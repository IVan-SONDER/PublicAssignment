\documentclass[12pt]{article}
\usepackage[utf8]{inputenc}
\usepackage{geometry}
\geometry{a4paper, margin=1in}
\usepackage{graphicx}
\usepackage{amsmath,amsfonts,amssymb}

\title{Assignment 1}
\author{Shuao Chen}
\date{\today}

\begin{document}

\maketitle
\begin{enumerate}
    \item Please list (at least) eight skills facilitating research.
    
       Eight essential skills that significantly enhance research effectiveness are as follows: Firstly, identifying and utilizing pertinent sources while selecting valuable and under-explored research topics. Secondly, skillfully narrowing down broad areas to pinpoint specific, less-researched topics. Thirdly, defining practical issues or problems with clarity and precision. Fourthly, elucidating the motivation behind solving these problems and the potential contributions that solutions may offer. Fifthly, designing innovative methods, frameworks, or algorithms to address these problems. Sixthly, exercising diligence in writing research papers, ensuring strict adherence to academic standards and guidelines. Seventhly, demonstrating rigorous reasoning coupled with comprehensive simulations or experimental evidence. Finally, undertaking thorough and conscientious revisions of the research paper to refine and perfect its content.

    \item How would you like to explore a new field? Please list (at least) three tips/method.
    
       To delve into a new field, I would employ a multi-step approach. Initially, I would immerse myself in survey papers to gain a broad understanding of the field. This would be followed by systematically gathering essential knowledge prevalent in the area. Subsequently, I would develop a structured schedule dedicated to intensive learning, aimed at strengthening my foundational knowledge. Finally, after accumulating sufficient experience, I would embark on my inaugural project in the new field, applying the insights and skills acquired during my preparatory phase.

    \item Please list (at least) four online databases for research.
    
       Four notable online databases invaluable for research are CIFAR10, MNIST, IMAGENET2012, and COCO. Each of these databases offers a unique repository of data, beneficial for various types of research, especially in the fields of machine learning and computer vision. CIFAR10 and MNIST are pivotal for benchmarking in image recognition tasks, while IMAGENET2012 provides a more diverse and challenging dataset. COCO, on the other hand, is renowned for its extensive dataset for object detection, segmentation, and captioning, making it indispensable for advanced research in these areas.

    \item Please share an example (not from the slides) of finding a research niche.
    
       An illustrative example of finding a research niche outside of the provided slides is the exploration of virtual reality (VR) technology. In VR, the challenges of handling massive data streams and real-time transmission requirements are prominent. To address these issues, it's crucial to introduce or develop novel techniques aimed at reducing the volume of transmitted content or expanding transmission pathways. Such a niche not only addresses a practical need but also propels forward the technological advancements in VR.

    \item What's your opinion on the difference between practical issues and theoretical problems? (in 200 words)
    
       The relationship between practical issues and theoretical problems is symbiotic and crucial for advancing any field. Theoretical problems provide a solid foundation and guidelines for addressing practical issues. For instance, Shannon's information theory has been instrumental in the development of efficient coding techniques like LDPC and Polar codes, enhancing real-world communication systems. Conversely, practical challenges often direct the focus of theoretical research, helping to identify which problems require solutions for field advancement. For example, the quest for effective data transmission in real-world applications has spurred the growth of source coding and channel coding theories.
       
\end{enumerate}
\end{document}
