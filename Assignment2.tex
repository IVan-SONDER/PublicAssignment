\documentclass[12pt]{article}
\usepackage[utf8]{inputenc}
\usepackage{geometry}
\geometry{a4paper, margin=1in}
\usepackage{graphicx}
\usepackage{amsmath,amsfonts,amssymb}

\title{Assignment 2}
\author{Shuao Chen}
\date{\today}

\begin{document}

\maketitle
\begin{enumerate}
    \item BDCAE

    \item ABAABBABA

    \item Please recommend an academic paper in your field and answer the following questions.
    
    \begin{enumerate}
        \item Please list the structure of the recommended paper (such as Introduction - Related Work - Method - Experiments - Discussion - Conclusion). Does this paper follow the hamburger-like structure?
        
           The structure of the academic paper titled "Intelligent Reflecting Surface (IRS)-Aided Covert Wireless Communications with Delay Constraint" adheres to a methodical and comprehensive hamburger-like structure. The paper begins with an introduction, which is meticulously segmented into background context, a review of related works, and a detailed outline of the contributions made by this study. This is followed by the main body of the paper, which meticulously delves into the system model, providing a detailed demonstration, thorough explanation, and vivid illustrations to support the theoretical underpinnings and practical applications of the research. The paper culminates with a well-articulated conclusion, summarizing the key findings and implications of the study.
        
        \item Please give a case of how the evidence is incorporated in this paper?
        
           The integration of evidence in this paper is executed with remarkable academic rigor and precision. For instance, the authors paraphrase existing literature to assert that the secrecy performance of IRS-assisted networks can be enhanced by optimally designing IRS reflection coefficients. This simultaneously strengthens the signal for intended users while attenuating it for eavesdroppers, as evidenced by references such as [16]. Additionally, the paper synthesizes various studies, including references [24]–[36], to highlight the challenges in achieving high covert communication rates due to stringent covertness requirements. The authors skillfully argue that the IRS's capability to enhance signals for legitimate receivers and degrade them for wardens is a practical and promising solution, citing a recent magazine article [37] and a study [38] to bolster their argument.
        
        \item Please write an abstract of this paper in your own words.
        
        In my own words: This paper presents an in-depth exploration of the use of Intelligent Reflecting Surfaces (IRS) in enhancing the efficiency and security of covert wireless communications. The research meticulously examines how the strategic adjustment of IRS reflection coefficients can significantly improve the secrecy performance in IRS-assisted networks. By doing so, it not only amplifies the signal reception for intended users but also minimizes the risk of eavesdropping. The study also addresses the inherent challenges in maintaining high rates of covert communication, especially given the strict demands for secrecy. The paper innovatively proposes the use of IRS to balance signal strength between legitimate receivers and potential eavesdroppers, making a compelling case for its practical application in secure wireless communication. This argument is further reinforced with references to contemporary studies and articles, demonstrating the cutting-edge nature and relevance of this research.
        
        The original one: This work examines the performance gain achieved by deploying an intelligent reflecting surface (IRS) in covert communications. To this end, we formulate the joint design of the transmit power and the IRS reflection coefficients by taking into account the communication covertness for the cases with global channel state information (CSI) and without a warden’s instantaneous CSI. For the case of global CSI, we first prove that perfect covertness is achievable with the aid of the IRS even for a single-antenna transmitter, which is impossible without an IRS. Then, we develop a penalty successive convex approximation (PSCA) algorithm to tackle the design problem. Considering the high complexity of the PSCA algorithm, we further propose a low-complexity two-stage algorithm, where analytical expressions for the transmit power and the IRS’s reflection coefficients are derived. For the case without the warden’s instantaneous CSI, we first derive the covertness constraint analytically facilitating the optimal phase shift design. Then, we consider three hardware related constraints on the IRS’s reflection amplitudes and determine their optimal designs together with the optimal transmit power. Our examination shows that significant performance gain can be achieved by deploying an IRS into covert communications.
    \end{enumerate}
       
\end{enumerate}
\end{document}
